\documentclass[a4paper, 12pt]{report}

\usepackage{graphicx}
\usepackage{textcomp}
\usepackage[italian]{babel}
\usepackage[a4paper, total={17cm, 24cm}]{geometry}
\usepackage{float}
\usepackage[font=small,format=plain,labelfont=bf,up,textfont=normal,up,justification=justified,singlelinecheck=false,skip=0.01\linewidth]{caption}
\renewcommand{\familydefault}{\sfdefault}
\linespread{1.1}

\title{Relazione progetto\\``Smart Dumpster''}
\author{Matteo Castellucci, Giorgia Rondinini}
\date{\today}

\begin{document}
	\maketitle
	\tableofcontents
	\chapter{Analisi dei requisiti}
	Si vuole realizzare un sistema di deposito rifiuti intelligente denominato ``Smart Dumpster".\newline
	Questo sistema dovrà permettere all'utente di gestire il deposito dei rifiuti tramite un'app Android,
	previo l'ottenimento di un codice speciale denominato ``token" da un componente, in esecuzione su un
	\textit{host} separato dal bidone fisico, definito ``Service". Una volta ottenuto il token occorrerà
	recarsi nei pressi del bidone, e collegarsi tramite Bluetooth alla componente ``Controller" del
	sistema entro cinque minuti, pena la scadenza del token. Ad avvenuto collegamento, sarà
	possibile scegliere il tipo di rifiuto da depositare ed
	effettuare così richiesta per un deposito. L'app contatterà per primo il Service, notificandolo della
	richiesta e, se il deposito sarà considerato possibile - ovvero se il sistema è nello stato ``\textit{available}"
	e il token passato al Service è lo stesso fornito poco prima dal Service all'app -, esso procederà
	con il notificare a sua volta la componente ``Edge" del sistema. Se questa operazione va a buon fine, 
	si notificherà il Controller di voler effettuare un deposito fornendo il tipo di rifiuto che si
	desidera depositare scegliendolo tra tre tipi - A, B o C - e questo aprirà gli sportelli per permettere
	all'utente di depositare i rifiuti. L'utente
	avrà un minuto per effettuare un deposito, dopo il quale le porte si chiuderanno e il bidone
	incamererà i rifiuti depositati. A questo punto, la componente Controller invierà un messaggio di
	terminazione deposito all'applicazione, che a sua volta notificherà il Service. Nel caso in cui
	l'utente abbia inserito troppi rifiuti eccedendo il peso massimo che il bidone può contenere, la
	componente Edge si preoccuperà di notificare immediatamente il Service, inviandogli anche il peso caricato fino a
	quel punto, e questo si preoccuperà di impostare lo stato del sistema su ``\textit{unavailable}". Se questo è
	il caso, il Service non si preoccuperà di notificare la terminazione del deposito anche all'Edge 
	quando ne riceverà notizia dall'app,
	essendo per esso già terminato. In caso contrario lo farà, ottenendo dall'Edge il peso dei rifiuti
	inseriti dall'utente.\newline
	Il sistema presenterà anche un sito web denominato ``Dashboard", riservato agli operatori
	ambientali, che permetterà di controllare lo stato del sistema, ovvero di controllare lo stato del
	cassonetto (se ``\textit{available}'' o ``\textit{unavailable}''), il peso dei rifiuti contenuti nel bidone e il 
	numero di depositi effettuati in giornata. Dalla ``Dashboard'' sarà inoltre possibile richiedere separatamente il \textit{log},
	ovvero il numero di depositi e la quantità totale depositata negli ultimi cinque giorni, e
	forzare lo stato del sistema per ragioni di manutenzione ad ``\textit{available}" o
	``\textit{not available}" in qualsiasi momento, anche durante un deposito, tenendo conto del fatto
	che forzare lo stato del bidone a ``\textit{available}" significa anche averlo svuotato, operazione che 
	porta ad azzerare il peso correntemente contenuto al suo interno.
	\begin{figure}[H]
		\centering
		\includegraphics[height=\textheight]{"img/Sequence"}    
		\caption{Diagramma di sequenza con le interazioni tra le componenti del sistema}
	\end{figure}
	\begin{figure}[H]
		\centering
		\includegraphics[width=\textwidth]{"img/SystemStatechart"}    
		\caption{Diagramma di stato del sistema nel suo complesso}
	\end{figure}
	\chapter{Design}
		\section{Specifiche hardware e schemi dei circuiti}
		Gli stati ``\textit{available}" e ``\textit{not available}" sono fisicamente rappresentati sul sistema da due LED,
		uno verde e uno rosso. Il sensore di peso utilizzato durante il deposito è
		rappresentato da un potenziometro e, assieme ai due LED, è pilotato dalla componente Edge del
		sistema, realizzata su un SoC ESP8266 ``NodeMCU 1.0".\newline
		Ai tre tipi di rifiuti sono associati altrettanti LED verdi, prima del deposito viene acceso
		quello corrispondente al tipo di rifiuto scelto e non viene spento fino al termine dello stesso.
		L'apertura e la chiusura degli sportelli sono pilotate da un servomotore, che come i tre LED
		sarà collegato alla componente Controller del sistema. Questa possiederà anche un modulo per la
		comunicazione Bluetooth ``HC-05" e sarà implementata con un microcontrollore Arduino Uno.
		\begin{figure}[H]
			\centering
			\includegraphics[width=0.9\textwidth]{"img/HardwareSchemas"}    
			\caption{Schemi dei circuiti per le componenti Controller ed Edge}
		\end{figure}
		\section{Design architetturale}
			\subsection{Dashboard}
			Il sito che realizza la ``Dashboard" è realizzato mediante risorse statiche dotate
			della struttura di base, ma senza alcuna informazione dentro. I dati vengono inseriti e il
			comportamento messo in atto mediante chiamate asincrone effettuate dallo \textit{user agent}
			una volta che la pagina è totalmente caricata, permettendo perciò di non avere pagine web
			che necessitano di \textit{preprocessing} lato server.
			\begin{figure}[H]
				\centering
				\includegraphics[width=0.7\textwidth]{"img/DashboardStatechart"}    
				\caption{Diagramma di stato per la componente Dashboard}
			\end{figure}
			\subsection{Controller}
			Il sistema ``Controller" è costituito da due componenti fondamentali, una di \textit{Model}
			e una di \textit{Controller}, cercando di replicare il pattern MVC senza però una componente
			di \textit{View}. Il \textit{Controller} è realizzato usando il pattern ``Event-loop", dove il
			sistema resta in attesa di nuovi messaggi trasmessi mediante Bluetooth e, ad ogni iterazione del 
			loop, converte in eventi i messaggi ricevuti, cioè converte elementi di Model in
			elementi di Controller. A questo punto, si controlla se nella \textit{event queue} sono
			presenti eventi: nel caso si estraggono e si eseguono i relativi \textit{handler}, che
			conterranno il comportamento da eseguire in caso accada quello specifico evento. I
			messaggi che può ricevere Arduino sono o per iniziare il deposito per uno dei tre
			tipi di rifiuto oppure per richiedere un'estensione di tempo per il deposito.
			\begin{figure}[H]
				\centering
				\includegraphics[width=0.7\textwidth]{"img/ControllerStatechart"}    
				\caption{Diagramma di stato per la componente Controller}
			\end{figure}
			\subsection{App}
			L'applicazione è costituita da una sola \emph{Activity} che possiede i bottoni necessari per
			avviare tutte le azioni che l'utente deve essere in grado di compiere tramite essa. Questi vengono abilitati e
			disabilitati mano a mano che si passa attraverso le varie fasi del deposito di rifiuti.
			Inizialmente è infatti possibile solamente connettersi ad un bidone via Bluetooth o
			richiedere un token con i due bottoni a disposizione. Quando si è connessi e si ha anche
			ricevuto un token l'app abilita i bottoni per scegliere il tipo di rifiuto e
			conseguentemente far partire il deposito, disabilitando i precedenti. Premuto uno qualsiasi
			dei tre bottoni questi vengono disabilitati, il Service viene informato della
			richiesta e, se tutto è andato a buon fine, si comunica la stessa cosa al Controller,
			fornendogli anche il tipo di rifiuto scelto. Il deposito può così iniziare, il pulsante 
			per l'estensione del tempo di deposito  viene abilitato e l'app inizia a inviare periodicamente 
			al Service dei messaggi di ``keep alive'' per segnalargli che il deposito è ancora in corso. Al termine del tempo, l'app 
			si preoccuperà di comunicare al Service la terminazione del deposito, disabilitando il pulsante 
			per l'estensione del tempo di apertura del cassonetto e tornando così alla configurazione iniziale 
			della GUI dell'app, eventualmente lasciando disabilitato il pulsante per connettersi via Bluetooth 
			al cassonetto se si è ancora connessi.
			\begin{figure}[H]
				\centering
				\includegraphics[width=1.03\textwidth]{"img/AppStatechart"}    
				\caption{Diagramma di stato per l'applicazione}
			\end{figure}
			\subsection{Edge}
			Il sistema "Edge" del sistema è anch'esso costituito da due componenti fondamentali, una di
			\textit{Model} e una di \textit{Controller}, rispecchiando una versione modificata del
			pattern MVC senza la componente di \textit{View}.\newline
			Il Controller è stato realizzato mediante il pattern ``Event-loop", dove ad ogni iterazione
			del ciclo di esecuzione principale si controlla se sono arrivati nuovi messaggi sotto forma
			di richieste HTTP al Web Server che esegue all'interno del SoC. Se sono presenti connessioni in attesa
			da parte dell'unico possibile client, il Service, queste vengono accettate e si
			estraggono i dati dalle richieste HTTP ricevute su esse per formare oggetti di tipo Message. Questi vengono poi, uno per
			iterazione, trasformati dall'Event-loop in eventi ed affidati al cosiddetto ``EventHandlerManager". 
			Questo si preoccupa di metterli in una coda di eventi che è il cuore
			del pattern ``Event-loop". Dopodiché, si estrae un evento per iterazione dalla coda e lo si
			riaffida all'EventHandlerManager affinché esegua il comportamento associato a
			quell'evento, ovverosia il suo \textit{handler}. In questo modo l'Event-loop è a conoscenza
			di quale evento viene gestito ad ogni iterazione.\newline
			I possibili eventi che
			l'Edge riesce a gestire sono il passaggio allo stato ``\textit{unavailable}'' e quello allo
			stato ``\textit{available}", dove quest'ultimo presenta la particolarità di azzerare il peso
			correntemente contenuto nel bidone perché corrisponde all'azione di svuotamento del bidone
			stesso e alla sua rimessa in funzione. Questi due cambi di stato possono essere effettuati
			in qualsiasi momento, anche durante il deposito.
			Oltre a ciò, è possibile comunicare ad Edge l'inizio
			di un deposito, cosa che verrà fatta esclusivamente se lo stato corrente è pari a
			``\textit{available}", e la terminazione dello stesso. Un deposito potrebbe terminare anche nel
			caso l'utente che sta effettuando il deposito appoggi troppo peso rispetto a quanto ne può
			ancora contenere il bidone. In ogni caso, al termine di una procedura di deposito viene
			segnalato il peso dei rifiuti inseriti. Da ultimo, ma non per questo meno importante, è
			possibile richiedere informazioni sullo stato corrente di Edge, come il peso correntemente
			contenuto e se esso è nello stato ``\textit{available}" o ``\textit{unavailable}".\newline
			\begin{figure}[H]
				\centering
				\includegraphics[width=0.8\textwidth]{"img/EdgeStatechart"}    
				\caption{Diagramma di stato per la componente Edge}
			\end{figure}
			\subsection{Service}
			Il sistema Service può essere visto come il \textit{digital twin} del bidone, ovvero il suo
			doppio virtuale che permette di estrapolare informazioni da esso e fornire servizi. Poiché
			può comunicare via Internet, usando HTTP, con i vari componenti del sistema, esso è composto
			sia di un web client che di un web server, nonché di una rappresentazione interna del
			componente Edge, la componente propriamente del bidone con cui può direttamente comunicare. Lo
			stato di questa viene mantenuto sempre allineato con il componente fisico, così da poter
			effettuare alcune operazioni più rapidamente, senza bisogno di connettersi tutte le volte
			a quest'ultimo. Service, dialogando con la Dashboard - la quale è di fatto il suo
			\textit{frontend} -, deve poter fornire informazioni aggiornate sullo stato corrente del
			bidone - peso dei rifiuti correntemente contenuti, numero di depositi effettuati, lo stato
			vero e proprio del bidone -, ma anche il \textit{log} dei precedenti cinque giorni, che
			contiene informazioni in forma riassunta sui depositi effettuati in passato. Inoltre, offre
			la possibilità di forzare lo stato ad ``\textit{available}" o ``\textit{unavailable}", 
			operazione che consiste semplicemente nell'inoltrare la richiesta al sistema Edge e 
			controllare successo o fallimento dell'operazione.\newline 
			Dialogando con l'app, Service permette di richiedere l'inizio o la fine di un
			deposito attraverso l'uso di un token così come descritto nelle specifiche. In caso
			l'applicazione notificasse la terminazione del deposito dopo che questo è già stato terminato
			da Edge perché l'utente ha raggiunto il peso massimo consentito di rifiuti per il bidone,
			Service si premura di non notificare nuovamente l'evento ad Edge, cosa che viene fatta invece se 
			Edge crede che il deposito sia ancora in corso.
			\begin{figure}[H]
				\centering
				\includegraphics[width=\textwidth]{"img/ServiceStatechart"}    
				\caption{Diagramma di stato per la componente Service. A sinistra si trova il diagramma riferito a Service, 
					a destra si trova il diagramma di stato della componente Edge, a cui Service fa riferimento}
			\end{figure}
		\section{Design dettagliato}
			\subsection{Dashboard}
			Il sito web della ``Dashboard" è realizzato interamente in HTML, CSS e Javascript. È stata
			utilizzata anche la libreria jQuery per semplificare l'uso di Javascript. Di questa in
			particolare è stata usata l'API per l'uso di AJAX, sfruttato per effettuare richieste
			asincrone al server ``Service" per ottenere dati sul bidone in formato JSON, cioè lo stato corrente,
			il peso dei rifiuti correntemente nel bidone, il numero di depositi effettuati e il log,
			oppure per permettere la gestione del comportamento del sistema, cioè le
			funzionalità di forzatura dello stato ``\textit{available}" o ``\textit{unavailable}".
			\subsection{Controller}
			Il sistema Arduino presenta un \textit{PhysicalSystem}, che rappresenta il sistema fisico, 
			un \textit{CommunicationSystem}, usato per incapsulare la gestione della comunicazione via Bluetooth, 
			e un sistema di gestione logica degli eventi e dei loro \textit{handler}.\newline
			L'utilizzo del Bluetooth è permesso dalla libreria ``SoftwareSerial", che permette di inviare
			e ricevere dati da un qualsiasi componente che scambia dati tramite connessione seriale
			asincrona come se si stesse usando la comunicazione seriale standard fornita da Arduino.\newline 
			I messaggi sono costituiti da un solo carattere e perciò, avendo una struttura così
			rigida, non è necessario l'uso di terminatori di messaggio (presenti però per come la comunicazione
			Bluetooth è gestita dall'applicazione, essi vengono ignorati).\newline
			In caso non sia presente alcun messaggio, l'oggetto che si occupa della comunicazione via
			Bluetooth restituisce un messaggio vuoto. Questo viene poi a sua volta convertito dal
			\textit{parser} in un evento nullo, a cui non è associato nessun \textit{handler} e che
			perciò non fa accadere nulla, in applicazione del ``Null Object" pattern.
			\subsection{App}
			Per la comunicazione via Bluetooth è stato deciso di utilizzare la libreria fornitaci dal
			professore senza applicare alcuna modifica. Inoltre, si è presupposto che il \textit{pairing}
			tra i due \textit{device}, App e Controller, sia avvenuto precedentemente e perciò basti
			stabilire una connessione tra i due.\newline
			Per la comunicazione con il ``Service'' sono state usate le librerie base di Android, cercando 
			di fattorizzare il codice per la creazione e l'invio dei messaggi, dato che essi hanno tutti 
			struttura simile e la stessa destinazione. La creazione dei messaggi avviene tramite un ``ServiceMessageBuilder'', 
			tramite cui devono essere inizializzati i parametri necessari al tipo di messaggio che si desidera
			costruire, mentre il loro invio è gestito da una ``AsyncTask'', per permettere all'utente di continuare 
			a interagire con l'app mentre si cerca di comunicare con il ``Service''. Per ogni messaggio da inviare 
			vengono usati un nuovo ``ServiceMessageBuilder'' e una nuova ``AsyncTask''.\newline
			Per impedire che il sistema si bloccasse per tempi indefiniti nel caso il ``Service'' non fosse 
			contattabile si è posto un limite alla durata della connessione HTTP.
			\subsection{Edge}
			La componente Model è suddivisa in tre sotto-componenti: una che contiene le classi che
			rappresentano i componenti fisici con cui ESP va ad interagire, una per le classi che
			contengono la logica di funzionamento dell'intero sistema, secondo un pattern di tipo
			``Strategy", che altro non sono che sotto-classi dell'interfaccia ``Handler", quindi 
			\textit{handler} di eventi specifici, e una che contiene tutte le classi per rappresentare
			all'interno di Edge la componente Service e la possibilità di scambiare messaggi con essa.
			Edge perciò, potendo sia inviare che ricevere messaggi, ed essendo questi scambiati via
			HTTP, deve poter fare sia da \textit{server} che da \textit{client} HTTP. In più, un
			messaggio è sia una \textit{request} HTTP che una \textit{response}, e quindi i metodi per lo
			scambio di messaggi dovranno distinguere nella creazione di oggetti di tipo ``Message" tra
			questi due tipi di messaggi HTTP. In realtà, poiché l'unico messaggio che Edge può inviare
			è quello per la terminazione di un deposito, la \textit{response} che Service fornisce per
			questo è poco significativa, perciò si è optato per ignorarla e non far gestire i
			messaggi di \textit{response} HTTP dal metodo di ricezione dei messaggi della classe Service.\newline 
			La componente di Controller, invece, possiede la classe ``EventScheduler" che fa da madre a tutte le
			altre, servendosi di ``EventHandlerManager" che trattiene la
			coda degli eventi nonché l'insieme degli \textit{handler} eseguibili e permette le operazioni
			su questi. È infatti possibile accodare e togliere dalla coda degli eventi uno specifico
			evento, inoltre è possibile aggiungere un handler, ma non rimuoverlo, dato che non c'era
			necessità di offrire questa funzionalità, visto che quali e quanti sono questi viene deciso
			a priori.\newline
			Per utilizzare l'astrazione della lista, anche per poterla gestire come se fosse una coda,
			anziché usare le strutture dati \textit{array} già presenti nel linguaggio si è fatto
			ricorso alla libreria ``LinkedList". Per il \textit{parsing} e l'\textit{encoding} dei dati in formato
			JSON ricevuti ed inviati si è fatto ricorso alla libreria ``ArduinoJSON". Per l'utilizzo di
			un \textit{client} e un \textit{server} web si è fatto ricorso alle
			librerie ``ESP8266WiFi", ``ESP8266HTTPClient" e ``ESP8266WebServer".
			\subsection{Service}
			Per realizzare l'applicazione ``Service'' ci si è appoggiati al \textit{framework} ``VertX" per Java
			che fornisce già tutti gli strumenti necessari per gestire un \textit{web server}, un
			\textit{web client} e per effettuare \textit{request} o formulare \textit{response} HTTP.
			È presente inoltre una classe che rappresenta una copia, mantenuta allineata con la versione fisica, del sistema
			Edge di cui Service è il \textit{digital twin}. L'oggetto prodotto da questa classe viene
			interpellato per quanto riguarda le operazioni di ottenimento dello stato corrente e viene
			aggiornato in caso di forzatura dello stato dalla Dashboard o da un deposito terminato
			anticipatamente e segnalato perciò da Edge. Informazioni come il conteggio corrente dei
			depositi giornalieri e la quantità totale di rifiuti gettati in uno specifico bidone
			viene salvata in degli appositi file su disco, uno per giorno, che vengono poi utilizzati
			per ottenere le informazioni per costruire il \textit{log}. Non appena un'operazione ne ha
			bisogno questi vengono creati, se non presenti, ma con entrambi i parametri citati
			posti a zero, così dall'operazione successiva saranno già pronti per l'uso. Per quanto
			riguarda la loro cancellazione, questa non è prevista in Service, dovrà essere l'operatore
			a cancellarli manualmente quando non saranno più necessari.
\end{document} 
